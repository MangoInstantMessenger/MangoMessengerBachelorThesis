In November 2014, the Electronic Frontier Foundation [\cite{von2010electronic}] listed seven traits that contribute to
the security of instant messengers:
\begin{itemize}
    \item Having communications encrypted in transit between all the links in the communication path.
    \item Having communications encrypted with keys the provider does not have access to (end-to-end encryption).
    \item Making it possible for users to independently verify their correspondent's identity by comparing key fingerprints.
    \item Having past communications secure if the encryption keys are stolen (forward secrecy).
    \item Having the source code open to independent review (open source).
    \item Having the software's security designs well-documented.
    \item Having a recent independent security audit.
\end{itemize}
In addition, the security of instant messengers may further be improved if they:
\begin{itemize}
    \item Do not log or store any information regarding any message or its contents.
    \item Do not log or store any information regarding any session or event.
    \item Do not rely on a central authority for the relaying of messages (decentralized computing).
\end{itemize}