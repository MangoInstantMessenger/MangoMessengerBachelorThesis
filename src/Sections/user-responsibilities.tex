User responsibilities and procedures are as follows:
\begin{itemize}
    \item \textbf{Ensure password strength.} Ensure that your instant messaging system account password meets Carnegie Mellon
    University [\cite{shay2010encountering}] recommendations for strong passwords.
    Refer to the guidelines for password management and to the managing your password web pages.
    \item \textbf{Keep updated.} Download and install security upgrades from instant messaging system companies.
    This software is frequently updated to address security flaws.
    \item \textbf{Prefer automatic updates.} Turn on automatic updates for your instant messaging program and install
    updates as soon as they are available.
    \item \textbf{Stay encrypted.} Investigate encryption for your instant messaging client.
    The Electronic Frontier Foundation provides Instant Messaging encryption resources.
    \item \textbf{Do not "remember" password.} Don't allow your Instant Messaging program to "remember" your password
    or automatically sign in to your account.
    \item \textbf{Check your connections.} Don't automatically accept incoming messages from sign-in names that are
    not on your contact list.
    If someone wants to begin to communicate with you via Instant Messaging System,
    they should email you or phone you to exchange Instant Messaging sign-in names.
    \item \textbf{File transfers.} Don't accept file transfers under any circumstances.
    File transfers are an easy way for hackers to launch virus attacks and are not scanned for viruses before reaching
    your computer.
    In this case, sending an attachment via e-mail would be a better alternative because you
    \begin{enumerate}
        \item Expect the communication
        \item The attachment will be scanned at the mail server in addition to the anti-virus application on your computer
    \end{enumerate}
    \item \textbf{Do not click links.} Don't click links sent to you in a message, even if they appear to be from
    someone you know.
    Many links often go to a site hosting malware or may be malformed in such a way as to exploit another vulnerability.
    \item \textbf{Protect Privacy of Sensitive Data.}
    Don't discuss via Instant Messaging System or install an Instant Messaging application on a computer containing
    sensitive data.
    Don't assume that your Instant Messaging conversations are private or secure.
    Most Instant Messaging programs are not encrypted.
    Therefore, someone listening on the network can read anything said in your Instant Messaging conversation.
    \item \textbf{Avoid file-sharing.}
    File-sharing increases the risk that unauthorized parties could gain access to the computer.
    \item \textbf{Stay virus protected.} Implement Virus Protection that includes network desktop and laptop solutions to handle both Instant Messaging System
    methods of delivery (Server Broker and Server Proxy).
\end{itemize}