Instant messaging systems achieve a great success and became the main mean of communication
between people via an internet.
Thanks to the simplicity and quickness of the message exchanging more and more people over the world start to use
instant messengers on daily basis.
However, such a great attention forces us to discuss another aspect of these systems, an aspect of the
information security and user privacy.
The high attention and wide usage of the instant messaging systems in both, commercial and non-commercial ways to be
a justification for selecting the subject.
The subject matter of current thesis is entire communication structure of IMS including cryptography, protocols,
data storage, means of communications.
As an object of research we consider the entire entity defined as instant messaging system, in context of modern world.
Mainly, the research is done using qualitative data gathered from various sources, which listed in the references.
We consider qualitative research as most suitable since that problem of security in IMS is quite classic and widely
discussed in scientific community.
Finally, we design and implement an instant messaging system that copes with the required functionalities and satisfies
the defined security requirements, considering previous research.