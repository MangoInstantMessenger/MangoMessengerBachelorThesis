% Explanation of key concepts of the problem we are solving, technologies, methods app using, 'why app needs that functionalities?'
% How this apps used in business and society.
%

In the aeon of an internet instant messengers are leading instruments of communication between people in business,
commercials, personal and many other aspects.
However, it is very important to overview the phenomenon of an instant messaging systems from every side of it.
Modern comfortable and quick means of communication may cost a lot for unattended business
as well as may act negatively to naive person using IMS first time.
Recent data leaks of 533 million Facebook users [\cite{holmes2021533}]
yet again warns us on potential privacy issues in IMS\@.
Instant messaging system is under many of dangers, from the first glance there are: phishing, virus spread,
insecure communication protocols, wrong setup of host's SSL certificate, missing end-to-end encryption etc.
Although, these issues are looking quite important in terms of user privacy, the most user's danger is user himself.
Bruce Schneier's famous citation
\begin{quote}
    \textit{Amateurs hack systems, professionals hack people}
\end{quote}
reflects an entire power of social engineering approach [\cite{luo2011social}], which is quite succeeded nowadays.
Therefore, it is always worth to keep an eye on the following recommendation for both, commercial and non-commercial
usages of an IMS
\begin{itemize}
    \item \textbf{Ensure password strength.} Ensure that your instant messaging system account password meets Carnegie Mellon
    University [\cite{shay2010encountering}] recommendations for strong passwords.
    Refer to the guidelines for password management and to the managing your password web pages.
    \item \textbf{Keep updated.} Download and install security upgrades from instant messaging system companies.
    This software is frequently updated to address security flaws.
    \item \textbf{Prefer automatic updates.} Turn on automatic updates for your instant messaging program and install
    updates as soon as they are available.
    \item \textbf{Stay encrypted.} Investigate encryption for your instant messaging client.
    The Electronic Frontier Foundation provides Instant Messaging encryption resources.
    \item \textbf{Do not "remember" password.} Don't allow your Instant Messaging program to "remember" your password
    or automatically sign in to your account.
    \item \textbf{Check your connections.} Don't automatically accept incoming messages from sign-in names that are
    not on your contact list.
    If someone wants to begin to communicate with you via Instant Messaging System,
    they should email you or phone you to exchange Instant Messaging sign-in names.
    \item \textbf{File transfers.} Don't accept file transfers under any circumstances.
    File transfers are an easy way for hackers to launch virus attacks and are not scanned for viruses before reaching
    your computer.
    In this case, sending an attachment via e-mail would be a better alternative because you
    \begin{enumerate}
        \item Expect the communication
        \item The attachment will be scanned at the mail server in addition to the anti-virus application on your computer
    \end{enumerate}
    \item \textbf{Do not click links.} Don't click links sent to you in a message, even if they appear to be from
    someone you know.
    Many links often go to a site hosting malware or may be malformed in such a way as to exploit another vulnerability.
    \item \textbf{Protect Privacy of Sensitive Data.}
    Don't discuss via Instant Messaging System or install an Instant Messaging application on a computer containing
    sensitive data.
    Don't assume that your Instant Messaging conversations are private or secure.
    Most Instant Messaging programs are not encrypted.
    Therefore, someone listening on the network can read anything said in your Instant Messaging conversation.
    \item \textbf{Avoid file-sharing.}
    File-sharing increases the risk that unauthorized parties could gain access to the computer.
    \item \textbf{Stay virus protected.} Implement Virus Protection that includes network desktop and laptop solutions to handle both Instant Messaging System
    methods of delivery (Server Broker and Server Proxy).
\end{itemize}

Beyond that, there are commercial usage recommendations of the IMS\@.
Researchers at [\cite{hindocha2003malicious}] conclude on the following aspects of the usage of IMS in enterprise
\begin{itemize}
    \item \textbf{Follow best security practices.}
    End users and corporations should employ basic security practices and products such as
    intrusion detection and antivirus to mitigate the risk.
    \item \textbf{Decide is it worth.}
    Corporations at the outset should assess whether instant messaging is even a business necessity.
    \item \textbf{Support.} Enterprise versions of
    the instant messaging products should be utilized and administrators should be on the lookout for
    future enterprise security solutions that specifically address instant messaging threats.
    \item \textbf{Run under VPN.} Enterprise instant message system should be served under VPN\@.
\end{itemize}