% Chapter 1

\chapter{Introduction}\label{ch:introduction}

%----------------------------------------------------------------------------------------

\newcommand{\keyword}[1]{\textbf{#1}}
\newcommand{\tabhead}[1]{\textbf{#1}}
\newcommand{\code}[1]{\texttt{#1}}
\newcommand{\file}[1]{\texttt{\bfseries#1}}
\newcommand{\option}[1]{\texttt{\itshape#1}}

%----------------------------------------------------------------------------------------


\section{General overview}\label{sec:general-overview}
In recent years, instant messaging systems have gained more and more popularity as a new means of communication over the Internet.
Instant messengers allow their users to exchange text messages but, unlike email, the sender and the recipient
of a message are online at the same time.
In this respect communicating via an instant messaging system is more similar to using telephone than mail.
Security is increasingly becoming an important issue.
People want to retain their privacy.
Communications should not be overheard, copied, blocked or modified by a third party.
However, the Internet is known to be weak and vulnerable with respect to privacy.
There is a considerable effort being made to incorporate security into the existing communication systems and to
create new secure communication tools.
Another important issue is scalability.
The scalability of a system is its ability to handle large numbers of users distributed over geographically
large areas without notably affecting the overall performance of the system.
With the growing popularity of the Internet and the increasing number of users, systems that have not been designed
to be scalable currently show some performance problems.
For example, some very popular Web services, such as the Polish site of a very popular TV program "Big Brother",
simply cannot handle all the requests of the people willing to access those pages.
There exist many instant messaging systems.\ The most popular ones, such as ICQ or MSN Messenger, can handle vast
numbers of users and are reasonably scalable.
However, they are reported to have major security flaws.
Some other instant messengers, such as Iris, implemented most often as research projects claim to be secure.
On the other hand, those systems suffer from scalability problems.
The main goal of my project was to create an instant messaging system which would be both secure and scalable.
Additionally, I wanted this system to have a reasonable set of functionalities.
In other words, we have to provide everything that is necessary to make it a convenient tool without overloading
it with all the bells and whistles that can be found in commercial applications.
The project was done in the context of Globe, a distributed system being developed at the Vrije Universiteit,
whose main concern is scalability.
My instant messenger uses one of the Globe services, namely the Location Service, to locate its users in the Internet.

\section{Mango messenger system functionality overview}\label{sec:general-im-functionality-overview}
Prior the implementation of any application, it is vital to define the main functionalities it will provide.
In this section, we define a set of the main functionalities provided by Mango Messenger system.

\subsection{User privacy service}\label{subsec:user-privacy-service}

See chapter Overall security problems

\subsection{Instant messaging service}\label{subsec:instant-messaging-service}

See chapter Instant messaging service

Chapter [number] gives a detailed overview of functional and non-functional requirements.

