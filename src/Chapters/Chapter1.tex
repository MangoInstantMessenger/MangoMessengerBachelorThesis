% Chapter 1

\chapter{Introduction}\label{ch:introduction}

%----------------------------------------------------------------------------------------

\newcommand{\keyword}[1]{\textbf{#1}}
\newcommand{\tabhead}[1]{\textbf{#1}}
\newcommand{\code}[1]{\texttt{#1}}
\newcommand{\file}[1]{\texttt{\bfseries#1}}
\newcommand{\option}[1]{\texttt{\itshape#1}}

%----------------------------------------------------------------------------------------


\section{General overview of IM systems}\label{sec:general-overview-of-im-system}
Nowadays, the Instant Messaging Systems (IMS) became the most widely-used and convenient way of communication between people via Internet.
These systems offer a simple and inexpensive way to continuance existing relationships and forming new by providing an attractive
means for sharing information and digital social interactions.
The quick development of IMS and the widening of their popularity sometimes moves the focus from possible security risks.
In the worst case, IMS exposes vulnerable to security and privacy channels to hackers and intruders [1] [2].
In existing IMS, there are multiple privacy and security issues that need to be resolved in order to protect user's confidential
information and shared data via these messaging applications [3].
Source [3] gives an analysis of Telegram Messenger and the related MTProto Protocol with cryptography
behind Telegram.
Moreover, an overview of current security status for some major IMS is provided.
Meanwhile, the researchers in [6] discussed types of threats on privacy of IMS and ranges of thereat effects for both, user and provider.
In this thesis, the most major security threats of IMS is described.
In order to reflect best practices we provide a prototype-application written as example.


\section{Security vulnerabilities of IM systems}\label{sec:security-vulnerabilities-of-im-systems}
There are numerous risks associated with the use of IM and as with any form of electronic communication one must take certain steps to mitigate those risks.
Such risks include:

\subsection{Revealing confidential information}\label{subsec:revealing-confidential-information}
Revealing confidential information over an unsecured delivery channel.
Public Instant Messaging transmits unencrypted information, so it should never be used for sensitive or confidential information.
The information is on the Internet and may be accessed by anyone.

\subsection{Spreading viruses and worms}\label{subsec:spreading-viruses-and-worms}
Instant Message (IM) programs are fast becoming a preferred method for launching network viruses and worms.
The lack of built-in security, the ability to download files and built-in "buddy list" of recipients create an
environment in which viruses and worms can spread quickly.
The threat is growing so fast that IM is quickly catching up to e-mail as a primary point of attack.

\subsection{Exposing the network to backdoor Trojans}\label{subsec:exposing-the-network-to-backdoor-trojans}

\subsection{Denial of Service Attacks}\label{subsec:denial-of-service-attacks}

\subsection{Hijacking Sessions}\label{subsec:hijacking-sessions}
Hijacking Sessions - Information received by IM is not authenticated.
There is no way to verify that a message really originated from the sender with whom the recipient
believes he or she is communicating during the session.
Chat sessions can be hijacked and users can be impersonated.

\subsection{Legal Liability resulting from downloading copyrighted materials}\label{subsec:legal-liability-resulting-from-downloading-copyrighted-materials}


