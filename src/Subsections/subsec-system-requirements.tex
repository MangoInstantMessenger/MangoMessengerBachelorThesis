Prior to software module implementation, it is essentially important to define the functionality module will obtain.
In this section we discuss functional and non-functional requirements of secure instant messaging system from customer's prospective.
Generally, there are three forms of software product requirements: business, functional, and non-functional.
Business requirements [\cite{dilworth2007creation}] typically answer how the product will address the needs of your company and its users.
They also reveal the business model of the app and what problems it can solve.
Functional requirements [\cite{malan2001functional}] are about functionalities that will be implemented in the application.
Non-functional requirements [\cite{chung2012non}] describe how these functionalities will be implemented.

Mostly common and simple way to define software product's functional requirements are User Stories.
User stories [\cite{cohn2004user}] should be understandable both to developers and to you as the client, and should be written in simple words.
The most popular way of writing a user story is with the following formula

\begin{center}
    \begin{spverbatim}
        "As a <user type>, I want <goal> so that <reason>."
    \end{spverbatim}
\end{center}

Now, let's group the main features of the application as follows

\begin{itemize}
    \item Registration
    \item Authentication
    \item Managing contacts
    \item Sending messages and media to individuals
    \item Creating and managing groups
    \item Sending messages and media to groups
    \item Viewing messages history
    \item Managing profile settings
    \item Navigation
\end{itemize}

In order not to overfill the document, an entire list of requirements, both functional and non-functional
are moved to the annexes.