\chapter{Impelentation}\label{ch:impelentation}


\section{Project tasks}\label{sec:project-tasks}
\begin{description}
    \item \hspace*{8mm}\textbf{Task 1.}\\
    \begin{tabular}{|p{0.5\textwidth}|p{0.5\textwidth}|}
        \hline
        Task name                           & \\
        \hline
        Entities involved to solve the task & \\
        \hline
        Task completion outcomes            & \\
        \hline
        Star date of task execution         & \\
        \hline
        End date of task execution          & \\
        \hline
    \end{tabular}
    \item \hspace*{8mm}\textbf{Task 2.}\\
    \begin{tabular}{|p{0.5\textwidth}|p{0.5\textwidth}|}
        \hline
        Task name                           & \\
        \hline
        Entities involved to solve the task & \\
        \hline
        Task completion outcomes            & \\
        \hline
        Star date of task execution         & \\
        \hline
        End date of task execution          & \\
        \hline
    \end{tabular}
    \item \hspace*{8mm}\textbf{Task 3.}\\
    \begin{tabular}{|p{0.5\textwidth}|p{0.5\textwidth}|}
        \hline
        Task name                           & \\
        \hline
        Entities involved to solve the task & \\
        \hline
        Task completion outcomes            & \\
        \hline
        Star date of task execution         & \\
        \hline
        End date of task execution          & \\
        \hline
    \end{tabular}
    \item \hspace*{8mm}\textbf{Task 4.}\\
    \begin{tabular}{|p{0.5\textwidth}|p{0.5\textwidth}|}
        \hline
        Task name                           & \\
        \hline
        Entities involved to solve the task & \\
        \hline
        Task completion outcomes            & \\
        \hline
        Star date of task execution         & \\
        \hline
        End date of task execution          & \\
        \hline
    \end{tabular}
\end{description}

%\textbf{Task 1} \\
%
%\begin{tabular}{|p{5cm}|p{7cm}|}
%    \hline
%    Task name                                                                            & \\
%    \hline
%    Entities involved in the fulfilment of the task (max: 5 sentences)                   & \\
%    \hline
%    Task completion outcomes (preparation / decisions / technical / dossier, etc. )      & \\
%    \hline
%    Star date of task execution                                                          & \\
%    \hline
%    End date of task execution                                                           & \\
%    \hline
%\end{tabular} \\
%
%\textbf{Task 2} \\
%
%\begin{tabular}{|p{5cm}|p{7cm}|}
%    \hline
%    Task name                                                                            & \\
%    \hline
%    Entities involved in the fulfilment of the task (max: 5 sentences)                   & \\
%    \hline
%    Task completion outcomes (preparation / decisions / technical / dossier, etc. )      & \\
%    \hline
%    Star date of task execution                                                          & \\
%    \hline
%    End date of task execution                                                           & \\
%    \hline
%\end{tabular}


\section{Project implementation}\label{sec:project-implementation}
[Please develop the theoretical assumptions of the project, including notes; present the empirical part of the project – research results and conclusions, as well as subject-matter description, etc.
Please present computations and calculations, if any, in annexes.
Please do not change the names of the points below.
There is no predefined structure within individual points.
Also, additional parts, forming individual points, can be enumerated according to your own concept.
The theoretical and empirical part should not exceed 50,000 characters.
Please use Times New Roman font, 12 pts, 1.5 spacing.] \\

\begin{enumerate}
    \item Theoretical assumptions
    \item Description of facts
    \item Empirical research
\end{enumerate}


\section{Project outcomes}\label{sec:project-outcomes}
[Please describe the achieved outcomes of the project.
If possible, please provide figures showing the described outcomes.
Please confront them with the objectives of the project.
This part should be between 2000 and 10,000 characters long.
Please use Times New Roman font, 12 pts, 1.5 spacing.
Full description of solutions that were worked out and project outcomes, if any, should be presented in annexes.]


\section{Usefulness of project}\label{sec:usefulness-of-project}
[Please justify how this project is useful (how the project can be used in practice).
The description should not exceed 6000 characters.
Please use Times New Roman font, 12 pts, 1.5 spacing.]


\section{Project self-evaluation}\label{sec:project-self-evaluation}
[Each of the project’s Authors describes his or her skills and competencies that were developed while working on the project and identifies issues encountered while working on the project.
If during the work on the project the team had not completed any tasks planned earlier, or omitted them altogether, please specify what were these tasks and why they had not been completed.
This part should not exceed 6000 characters.
Please use Times New Roman font, 12 pts, 1.5 spacing.]


\section{Material and bibliography used to carry out the project}\label{sec:material-and-bibliography-used-to-carry-out-the-project}
[Please enumerate sources used by the team during the work on the project (as per the applying layout, Times New Roman font, 12 pts., 1.5 spacing).]


\section{List of annexes}\label{sec:list-of-annexes}
[In this place you should list all additional documents, e.g. preprinted forms, data sets, financial statements, survey templates, diagrams,
    concepts, strategies, studies, analyses, procedures, regulations, technical documents, plans, models, etc. which significantly contributed to the project.
Please prepare all annexes in accordance with the template in place.
Please use Times New Roman font, 12 pts, 1.5 spacing.
All annexes form an integral part of the project.]

