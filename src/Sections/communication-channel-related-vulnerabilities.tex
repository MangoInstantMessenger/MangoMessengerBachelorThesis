\subsection{Exposing the network to backdoor Trojans}\label{subsec:exposing-the-network-to-backdoor-trojans}
Malware such as adware, spyware, worms, Trojans, and other viruses can easily be transmitted through the
Instant Messaging program.
This also includes phishing programs that disguise themselves as legitimate and then trick you into revealing
your personal information.

\subsection{Denial of Service Attacks}\label{subsec:denial-of-service-attacks}
In computing, a denial-of-service attack (DoS attack) is a cyber-attack in which the perpetrator seeks to make a machine or
network resource unavailable to its intended users by temporarily or indefinitely disrupting services of a host connected
to the Internet.
Denial of service is typically accomplished by flooding the targeted machine or resource with superfluous requests in
an attempt to overload systems and prevent some or all legitimate requests from being fulfilled, see~\cite{gu2007denial}.
In a distributed denial-of-service attack (DDoS attack), the incoming traffic flooding the victim originates from
many different sources.
This effectively makes it impossible to stop the attack simply by blocking a single source.
A DoS or DDoS attack is analogous to a group of people crowding the entry door of a shop, making it hard for legitimate
customers to enter, thus disrupting trade.

Criminal perpetrators of DoS attacks often target sites or services hosted on high-profile web servers such as banks or
credit card payment gateways.
Researches at~\cite{prince2016empty, halpin2012philosophy} conclude that revenge, blackmail and activism can
motivate these attacks.

\subsection{Hijacking Sessions}\label{subsec:hijacking-sessions}
In computer science, session hijacking, sometimes also known as cookie hijacking is the exploitation of a valid computer
session -- sometimes also called a session key to gain unauthorized access to information or services in a computer system.
In particular, it is used to refer to the theft of a magic cookie used to authenticate a user to a remote server.
It has particular relevance to web developers, as the HTTP cookies used to maintain a session on many web sites
can be easily stolen by an attacker using an intermediary computer or with access to the saved cookies on the victim's
computer.
After successfully stealing appropriate session cookies an adversary might use the Pass the Cookie technique to perform
session hijacking.
By~\cite{bugliesi2015cookiext}, the cookie hijacking is commonly used against client authentication on the internet.
Modern web browsers use cookie protection mechanisms to protect the web from being attacked.
A popular method is using source-routed IP packets.
This allows an attacker at point B on the network to participate in a conversation between A and C by encouraging the
IP packets to pass through B's machine.
If source-routing is turned off, the attacker can use "blind" hijacking, whereby it guesses the responses of the two
machines.
Thus, the attacker can send a command, but can never see the response.
However, a common command would be to set a password allowing access from elsewhere on the net.

An attacker can also be "inline" between A and C using a sniffing program to watch the conversation.
This is known as a "man-in-the-middle attack" [\cite{callegati2009man}].

\subsection{Cross-site request forgery}\label{subsec:cross-site-request-forgery}
Cross-site request forgery, also known as one-click attack or session riding and abbreviated as CSRF or XSRF,
is a type of malicious exploit of a website where unauthorized commands are submitted from a user that the web
application trusts [\cite{barth2008robust}].
There are many ways in which a malicious website can transmit such commands;
specially-crafted image tags, hidden forms, and JavaScript XMLHttpRequests, for example, can all work without
the user's interaction or even knowledge.
Unlike cross-site scripting (XSS), which exploits the trust a user has for a particular site,
CSRF exploits the trust that a site has in a user's browser.

In a CSRF attack, an innocent end user is tricked by an attacker into submitting a web request that they did not intend.
This may cause actions to be performed on the website that can include inadvertent client or server data leakage,
change of session state, or manipulation of an end user's account.

CSRF is also used as an abbreviation in defences against CSRF attacks, such as techniques that use header data,
form data, or cookies, to test for and prevent such attacks.

\subsection{Cross-site scripting}\label{subsec:cross-site-scripting}
Cross-site scripting (XSS) is a type of security vulnerability typically found in web applications [\cite{vogt2007cross}].
XSS attacks enable attackers to inject client-side scripts into web pages viewed by other users.
A cross-site scripting vulnerability may be used by attackers to bypass access controls such as the same-origin policy.
Cross-site scripting attacks use known vulnerabilities in web-based applications, their servers, or the plug-in systems
on which they rely.
Exploiting one of these, attackers fold malicious content into the content being delivered from the compromised site.
When the resulting combined content arrives at the client-side web browser, it has all been delivered from the trusted source,
and thus operates under the permissions granted to that system.
By finding ways of injecting malicious scripts into web pages, an attacker can gain elevated access-privileges
to sensitive page content, to session cookies, and to a variety of other information maintained by the browser on
behalf of the user.
Cross-site scripting attacks are a case of code injection.