\chapter{Azure deployment via GitHub Actions}\label{ch:azure-deloyment-via-github-actions}
In this attachment deployment process to Azure via GitHub pages is described.


\section{Creating Database}\label{sec:creating-database}


\section{Setting up Azure Virtual Machine}\label{sec:setting-up-azure-environment}
As for student, Azure offers a lot of 100USD limited "free" opportunities,
provided that we won't pay in the future.
We require only a few things: Ubuntu based virtual machine, database, static IP address.
Azure provides such discounts for the
\begin{itemize}
    \item Virtual machine \texttt{Standard B1s (1 vcpu, 1 GiB memory)}
    \item Database \texttt{Burstable, B1ms, 1 vCores, 2 GiB RAM, 32 GiB storage}
\end{itemize}
And that's enough for our purposes.
Approximate cost per month for \texttt{Standard B1s (1 vcpu, 1 GiB memory)} is 8,25 USD/month as for December 2021.
Let's create \texttt{Standard B1s (1 vcpu, 1 GiB memory)} virtual machine with the following parameters
\begin{itemize}
    \item VM name: MangoMessengerWSBProject-Backend
    \item Region: (Europe) North Europe
    \item Image: Ubuntu Server LTS 20.04 - Gen2
    \item Authentication type: SSH public key, paste key to the field
    \item Public inbound ports: 80, 443, 22
    \item OS disk type: Standard SSD
    \item IP Address: SKU -- Basic, Assignment -- Static
    \item Azure Security Center, Enable basic plan for free: disabled
    \item Monitoring, Boots diagnostics: disabled
\end{itemize}
Next, let's connect to our recently created VM via SSH, we use WSL2 Ubuntu 20.04 under Windows.
The command to connect as follows

\begin{spverbatim}
    ssh -i /home/pkolosov/.ssh/id_rsa razumovsky_r@IP_ADDRESS
\end{spverbatim}

Therefore, we have connected successfully.
Let's install \texttt{nginx} server following commands under SSH connection on target VM
\begin{itemize}
    \item \texttt{sudo apt update -y}
    \item \texttt{sudo apt install -y nginx build-essential}
\end{itemize}
Now let's build our backend project, provided we work under WSL2 Ubuntu 20.04.
Prior to the build, we have to install dotnet build tools, see
\href{https://docs.microsoft.com/en-us/dotnet/core/install/linux-ubuntu}{MSDN}.
To build project and copy project to remote server,
navigate to its folder under WSL, then execute

\begin{itemize}
    \item \texttt{dotnet publish "Project.csproj" -r linux-x64 -o /build/path}
    \item \texttt{scp -r -i "privateKey/path" "build/path" user@IP:"remote/path"}
\end{itemize}

Now we have to configure the service, which will be responsible to run application,
the code is as follows

\begin{center}
    \begin{spverbatim}
    [Unit]
        Description=Mango Messenger Backend Service
        After=network.target

        [Service]
        Environment=ASPNETCORE_URLS=http://+:8080/
        Environment=BACKEND_ADDRESS="https://backend.address/"
        Environment=DATABASE_URL="connection string"
        Environment=EMAIL_SENDER_ADDRESS="notifications email"
        Environment=EMAIL_SENDER_PASSWORD="password"
        Environment=FRONTEND_ADDRESS="https://frontend.address/"
        Environment=JWT_LIFETIME="5"
        Environment=MANGO_AUDIENCE="https://backend.address/"
        Environment=MANGO_ISSUER="https://frontend.address/"
        Environment=MANGO_TOKEN_KEY="token sign secret"
        Environment=REFRESH_TOKEN_LIFETIME="token lifetime integer"
        Environment=SEED_PASSWORD="seed users password"
        Type=simple
        WorkingDirectory=/home/razumovsky_r/mango-backend
        ExecStart=/home/razumovsky_r/mango-backend/MangoAPI.Presentation
        User=razumovsky_r
        Group=razumovsky_r

        [Install]
        WantedBy=multi-user.target
    \end{spverbatim}
\end{center}
Remains only to create service inside VM and copy code there

\begin{itemize}
    \item \texttt{sudo vim /etc/systemd/system/mangoback.service}
\end{itemize}

To start service use

\begin{center}
    \begin{spverbatim}
        sudo systemctl start mangoback
    \end{spverbatim}
\end{center}

Right after the call, migrations should be applied to database.

Let's validate the server response via get request using curl

\begin{center}
    \begin{spverbatim}
        curl -I http://localhost:8080/swagger/index.html
    \end{spverbatim}
\end{center}

To expose the server outside, we have to setup nginx server following way

\begin{center}
    %! language = Latex
    \begin{spverbatim}
        server {
            listen 80;
            server_name SERVER_IP;

            location / {
                include proxy_params;
                proxy_pass http://127.0.0.1:8080;
            }

            location /swagger {
                include proxy_params;
                proxy_pass http://127.0.0.1:8080;
            }

            location /api {
                include proxy_params;
                proxy_pass http://127.0.0.1:8080;
            }

            # used in realtime communication
            location /notify {
                proxy_pass http://127.0.0.1:8080;
                proxy_http_version 1.1;
                proxy_set_header Upgrade $http_upgrade;
                proxy_set_header Connection "upgrade";
                proxy_set_header Host $host;
                proxy_cache_bypass $http_upgrade;
            }
        }
    \end{spverbatim}
\end{center}

However, the data transmission won't be secured.
Next, we have to configure SSL certificate for our server.
Install the packages
\begin{itemize}
    \item \texttt{sudo apt update -y}
    \item \texttt{sudo apt install -y python3 python3-pip python3-dev build-essential}
    \item \texttt{sudo pip3 install --upgrade pip}
    \item \texttt{sudo pip3 install certbot}
\end{itemize}
