\chapter{Instant messaging system requirements}\label{ch:instant-messaging-system-requirements}

In previous sections we have briefly discussed Instant Messaging System, mainly from security and user privacy aspects.
Prior to software module implementation, it is essentially important to define the functionality module will obtain.
Not to overcomplicate this section, we discuss the secure IMS requirements from customer's prospective.

There are different types of product requirements: business, functional, and non-functional.
Business requirements typically answer how the product will address the needs of your company and its users.
They also reveal the business model of the app and what problems it can solve.
Functional requirements are about functionalities that will be implemented in the app.
Non-functional requirements describe how these functionalities will be implemented.

In this article, we only focus on functional requirements.
In simple words, functional requirements are not ideas of how to solve problems or which technologies to use but rather
are descriptions of software functionality.
Mostly common and simple way to define software product's functional requirements is User Stories.
User stories should be understandable both to developers and to you as the client, and should be written in simple words.
The most popular way of writing a user story is with the following formula:

\begin{center}
    "As a <user type>, I want <goal> so that <reason>."
\end{center}


\section{Functional requirements}\label{sec:functional-requirements}
To compete with successful and commonly used instant messaging platforms, your service has to offer great functionality.
So first, let’s define the core features of a messaging app.

\begin{itemize}
    \item Registration, Authorization, Authentication
    \item Sending messages and media to individuals
    \item Manage contacts
    \item Creating groups
    \item Sending messages and media to groups
    \item Viewing message history
\end{itemize}
Note that Authentication means confirming your own identity, whereas authorization means being allowed access to the system.

\subsection{Registration, Authorization, Authentication}
\begin{itemize}
    \item As an unregistered user, I want to tap “sign up” so that I can see the registration form.
    \item As an unregistered user, I want to use my phone number to register so that my account is tied to my phone number.
    \item As an unregistered user, I want to add a display name so that other users can find my account not only by my phone number.
    \item As an unregistered user, I want to receive the registration confirmation via SMS so that I can activate my account.
    \item As an unregistered user, I want to receive the registration confirmation via Email so that I can activate my account.
    \item As a registered user, I want to authorize myself using both combinations email-password and phone-password, so that I see the specified form with two inputs.
    \item As a registered user, I want to tap "logout", so that only current device will be logged out.
    \item As a registered user, I want to tap "logout all", so that all my authorized devices will be logged out.
    \item As a registered user, I want my session on each device to be 7 days, so that after 7 days of inactivity device will be logged out automatically.
\end{itemize}

\subsection{Manage contacts feature user stories}
\begin{itemize}
    \item As a registered user, I want to be able to add other users as contacts, so that each user profile has a dedicated button.
    \item As a registered user, I want to be able to blacklist other users and do not receive any messages and notification from them,
    so that each user profile has a dedicated button.
\end{itemize}

\subsection{Sending messages and media feature user stories}
\begin{itemize}
    \item As a registered user, I want to send a text message so that another user gets a notification and sees my message.
    \item As a user, I want to see the status of a sent message so that I know if it’s been seen.
\end{itemize}

\subsection{Creating groups feature user stories}
\begin{itemize}
    \item As a registered user, I want to start secret chats with users from my contact list so that we can send messages that stay only on our devices.
    \item As a registered user, I want my secret chats to be device-specific so that I can see a secret chat only on the device that I used to start this chat.
    \item As a member of a secret chat, I want my secret messages to be protected from forwarding so that secret messages stay in secret chats.
    \item As a member of a secret chat, I want to get a notification when another member of the secret chat takes a screenshot of it
\end{itemize}

\subsection{Sending messages and media to groups feature user stories}

\subsection{Viewing message history feature user stories}


\section{Non-functional requirements}\label{sec:non-functional-requirements}