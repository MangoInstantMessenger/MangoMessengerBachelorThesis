\subsection{Revealing confidential information}\label{subsec:revealing-confidential-information}
Revealing confidential information over an unsecured delivery channel.
Public Instant Messaging transmits unencrypted information, so it should never be used for sensitive or confidential
information.
The information is on the Internet and may be accessed by anyone.

\subsection{Spreading viruses and worms}\label{subsec:spreading-viruses-and-worms}
Instant Message programs are fast becoming a preferred method for launching network viruses and worms.
The lack of built-in security, the ability to download files and built-in "buddy list" of recipients create an
environment in which viruses and worms can spread quickly.
The threat is growing so fast that Instant Messenger is quickly catching up to e-mail as a primary point of attack.

\subsection{Copyright infringement}\label{subsec:copyright-infringement}
Copyright infringement [\cite{hardy2002criminal}] is the use of works protected by copyright law without
permission for a usage where such permission is required, thereby infringing certain exclusive rights granted to the
copyright holder, such as the right to reproduce, distribute, display or perform the protected work, or to make
derivative works.
The copyright holder is typically the work's creator, or a publisher or other business to whom copyright has been assigned.
Copyright holders routinely invoke legal and technological measures to prevent and penalize copyright infringement.
Copyright infringement disputes are usually resolved through direct negotiation, a notice and take down process, or
litigation in civil court.
Egregious or large-scale commercial infringement, especially when it involves counterfeiting, is sometimes prosecuted
via the criminal justice system.
Shifting public expectations, advances in digital technology and the increasing reach of the Internet have led to such
widespread, anonymous infringement that copyright-dependent industries now focus less on pursuing individuals who seek
and share copyright-protected content online, and more on expanding copyright law to recognize and
penalize, as indirect infringers, the service providers and software distributors who are said to facilitate and
encourage individual acts of infringement by others.
Estimates of the actual economic impact of copyright infringement vary widely and depend on other factors.
Nevertheless, copyright holders, industry representatives, and legislators have long characterized copyright
infringement as piracy or theft -- language which some US courts now regard as pejorative or otherwise contentious,
see~\cite{powell1984dowling, li2009intellectual}.