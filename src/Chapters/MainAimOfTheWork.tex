\chapter{Main aim of the work}\label{ch:main-aim-of-the-work}

The main aim of this thesis is to design and implement an instant messaging system
that copes with the required functionalities and satisfies the defined security requirements.
More precisely, the following steps
\begin{enumerate}
    \item To propose the functional and non-functional requirements for instant messaging system.
    \item To eliminate security and user privacy vulnerabilities of the instant messaging system.
    \item To propose traits of secure instant messenger application among with user responsibilities.
    \item To design and discuss instant messaging system
    \begin{itemize}
        \item To propose optimal web service architecture.
        \item To propose mitigations for security and user privacy vulnerabilities (2).
        \item To design database structure.
        \item To propose planned technologies list to be used during implementation of the system.
    \end{itemize}
    \item To discuss and propose the optimal authentication/authorization mechanism for instant messaging system.
    \item To discuss and apply E2E Encryption.
    \item To design user interface which fits the designated functional requirements.
    \item To implement following modules
    \begin{itemize}
        \item \textbf{Web Service (API).} Backend part of the project.
        Application Programming Interface that allows developers to create their own clients.
        Web service to be implemented using latest for the moment of writing of this thesis .NET 5 platform.
        \item \textbf{Web Client} –- Web version of the Mango Messenger.
        Web client to be implemented using Angular front-end framework with TypeScript programming language.
        \item \textbf{Mobile Client} –- Mobile version of the Mango Messenger for target platforms: Android, IOS\@.
        \item \textbf{Desktop Client} –- Desktop version of the Mango Messenger for target platforms Windows, Linux, MacOS\@.
        Desktop client to be implemented by means of existing web client and ElectronJS framework.
    \end{itemize}
\end{enumerate}