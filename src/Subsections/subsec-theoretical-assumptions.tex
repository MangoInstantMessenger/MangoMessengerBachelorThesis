% Explanation of key concepts of the problem we are solving, technologies, methods app using, 'why app needs that functionalities?'
% How this apps used in business and society.
%

In the aeon of an internet instant messengers are leading instruments of communication between people in business,
commercials, personal and many other aspects.
However, it is very important to overview the phenomenon of an instant messaging systems from every side of it.
Modern comfortable and quick means of communication may cost a lot for unattended business
as well as may act negatively to naive person using IMS first time.
Recent data leaks of 533 million Facebook users [\cite{holmes2021533}]
yet again warns us on potential privacy issues in IMS\@.
Instant messaging system is under many of dangers, from the first glance there are: phishing, virus spread,
insecure communication protocols, wrong setup of host's SSL certificate, missing end-to-end encryption etc.
Although, these issues are looking quite important in terms of user privacy, the most user's danger is user himself.
Bruce Schneier's famous citation
\begin{quote}
    \textit{Amateurs hack systems, professionals hack people}
\end{quote}
reflects an entire power of social engineering approach [\cite{luo2011social}], which is quite succeeded nowadays.
Therefore, it is always worth to keep an eye on the following recommendation for both, commercial and non-commercial
usages of an IMS
\begin{itemize}
    \item \textbf{Ensure password strength.} Make sure that instant messaging system account you use for has enough
    strong password such that meets Carnegie Mellon University [\cite{shay2010encountering}] recommendations
    for strong passwords.
    \item \textbf{Keep updated.} Always download and install an updates from your instant messenger provider,
    often such updates are about security and user privacy.
    \item \textbf{Prefer automatic updates.} Keep on automatic updates for your instant messaging client and install
    updates as soon as they are released.
    \item \textbf{Stay encrypted.} Conduct research on encryption system of IMS\@.
    Ensure that instant messenger you are using supports end-to-end encryption.
    \item \textbf{Do not "remember" password.} Keep off the feature "remember my password" of your
    instant messaging program.
    Prefer to log in and logout from the system manually.
    \item \textbf{Check your connections.} Do not accept a messages from strangers, often these are of spam or of stealing
    personal data.
    \item \textbf{File transfers.} Prefer to share files via an e-mail, not by means of your instant messenger.
    \item \textbf{Do not click links.} Do not click any random links under any circumstances, even you know the sender
    personally.
    Frequently, such links may lead to an infected web resources.
    \item \textbf{Protect Privacy of Sensitive Data.}
    Do not keep any private on sensitive data on the computer instant messenger installed on.
    Moreover, do not discuss any sensitive or private topics via instant messenger.
    Therefore, someone listening on the network can read anything said in your Instant Messaging conversation.
    \item \textbf{Stay virus protected.} Ensure properly implemented virus protection among with firewall rules
    on the target machine.
\end{itemize}

Beyond that, there are commercial usage recommendations of the IMS\@.
Researchers at [\cite{hindocha2003malicious}] conclude on the following aspects of the usage of IMS in enterprise
\begin{itemize}
    \item \textbf{Follow best security practices.}
    End users and corporations should employ basic security practices and products such as
    intrusion detection and antivirus to mitigate the risk.
    \item \textbf{Wager cons and props.}
    Corporations at the outset should assess whether instant messaging is even a business necessity.
    \item \textbf{Support.} Enterprise versions of
    the instant messaging products should be utilized and administrators should be on the lookout for
    future enterprise security solutions that specifically address instant messaging threats.
    \item \textbf{Run under VPN.} Enterprise instant message system should be served under VPN\@.
\end{itemize}