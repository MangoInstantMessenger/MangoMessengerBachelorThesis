\chapter{System Requirements}\label{ch:system-requirements}

In previous sections we have briefly discussed Instant Messaging System, mainly from security and user privacy aspects.
Prior to software module implementation, it is essentially important to define the functionality module will obtain.
Not to overcomplicate this section, we discuss the secure Instant Messaging Systems requirements from customer's prospective.

There are different types of product requirements: business, functional, and non-functional.
Business requirements [\cite{dilworth2007creation}] typically answer how the product will address the needs of your company and its users.
They also reveal the business model of the app and what problems it can solve.
Functional requirements [\cite{malan2001functional}] are about functionalities that will be implemented in the app.
Non-functional requirements [\cite{chung2012non}] describe how these functionalities will be implemented.

In this article, we only focus on functional requirements.
In simple words, functional requirements are not ideas of how to solve problems or which technologies to use but rather
are descriptions of software functionality.
Mostly common and simple way to define software product's functional requirements is User Stories.
User stories [\cite{cohn2004user}] should be understandable both to developers and to you as the client, and should be written in simple words.
The most popular way of writing a user story is with the following formula:

\begin{center}
    \begin{spverbatim}
        "As a <user type>, I want <goal> so that <reason>."
    \end{spverbatim}
\end{center}


\section{Functional Requirements}\label{sec:functional-requirements}
To compete with successful and commonly used instant messaging platforms, your service has to offer great functionality.
So first, let’s define the core features of a messaging app.
\begin{itemize}
    \item Registration
    \item Authentication
    \item Authorization
    \item Adding contacts
    \item Sending messages and media to individuals
    \item Creating groups
    \item Sending messages and media to groups user stories
    \item Viewing message history
    \item Profile settings
\end{itemize}
Note that Authentication [\cite{burrows1989logic}] means confirming your own identity,
whereas Authorization [\cite{fagin1978authorization}] means being allowed access to the particular part of the system.

\subsection{Registration user stories}\label{subsec:registration}
\begin{itemize}
    \item As an unregistered user, I want to tap “register” so that I can see the registration form.
    \item As an unregistered user, I want to use my phone number to register so that my account is tied to my phone number.
    \item As an unregistered user, I want to use my e-mail to register so that my account is tied to my phone number.
    \item As an unregistered user, I want to add a display name during registration so that other users can find
    my account not only by my phone number or e-mail.
    \item As an unregistered user, I want to choose how to receive the registration confirmation via SMS or e-mail
    so that notification is sent by SMS or e-email.
    \item As an unregistered user, I want to receive the registration confirmation via SMS or Email so that
    I can activate my account.
\end{itemize}

\subsection{Authentication user stories}\label{subsec:authentication-user-stories}
\begin{itemize}
    \item As an un-authenticated user, I want to authenticate myself using both combinations email-password
    and phone-password so that I use the specified form with two inputs.
    \item As an authenticated user, I want my session on each device to least 7 days
    so that after 7 days of inactivity device will be logged out automatically.
\end{itemize}

\subsection{Adding contacts user stories}\label{subsec:adding-contacts}
\begin{itemize}
    \item As an authorized user, I want to add other user to my contact list so that each user profile has a dedicated button.
    \item As an authorized user, I want to remove the user from my contact list so that each contact profile has a dedicated button.
    \item As an authorized user, I want to send message to the user from my contact list so that each contact profile has a dedicated button.
\end{itemize}

\subsection{Sending messages and media to individuals user stories}
\label{subsec:sending-messages-and-media-feature-user-stories}
\begin{itemize}
    \item As an authorized user, I want to send a text message so that another user sees my message.
    \item As an authorized user, I want to send a document so that another user sees the document I sent.
    \item As an authorized user, I want to tap "Edit" on my message, so that I edit the message sent by myself.
    \item As an authorized user, I want to tap "Delete" on my message, so that I delete the message sent by myself.
\end{itemize}

\subsection{Creating groups user stories}\label{subsec:creating-groups-feature-user-stories}
\begin{itemize}
    \item As a registered user, I want to tap "details" -> "create group" in sidebar so that create a new group.
    \item As a registered user, I want to tap "details" -> "new chat" in sidebar, so that create a new direct chat with specified user.
    \item As a registered user, I want to join public groups, so that there is a button "join" on chat layout.
    \item As a registered user, I want to start secret chats with users from my contact list so that we can send messages that stay only on our devices.
    \item As a registered user, I want my secret chats to be device-specific so that I can see a secret chat only on the device that I used to start this chat.
    \item As a member of a secret chat, I want my secret messages to be protected from forwarding so that secret messages stay in secret chats.
    \item As a member of a secret chat, I want to get a notification when another member of the secret chat takes a screenshot of it.
\end{itemize}

\subsection{Sending messages and media to groups user stories}
\label{subsec:sending-messages-and-media-to-groups}
\begin{itemize}
    \item As an authorized user, I want to send a text message so that all members of a group see my message.
    \item As an authorized user, I want to send a document so that all members of a group see the document I sent.
    \item As an authorized user, I want to tap "Edit" on my message, so that I edit the message sent by myself.
    \item As an authorized user, I want to tap "Delete" on my message, so that I delete the message sent by myself.
\end{itemize}

\subsection{Viewing message history user stories}\label{subsec:viewing-message-history-feature-user-stories}
\begin{itemize}
    \item As an authorized user, I want to be able to view a message history of particular chat or group
    so that I see a list of my active chats on the UI\@.
\end{itemize}

\subsection{Profile settings user stories}\label{subsec:profile-settings-user-stories}
\begin{itemize}
    \item As an authorized user, I want to be able to change my personal information so that I use a specified form.
    \item As an authorized user, I want reset password, so that my password will change.
    \item As an authorized user, I want to tap "Logout" button so that current device will be logged out from the system.
    \item As an authorized user, I want to tap "Logout all" button, so that all my authorized devices will be
    logged out from the system.
\end{itemize}



\section{Non-Functional Requirements}\label{sec:non-functional-requirements}
\begin{itemize}
    \item \textbf{NFR01.} The system must be enjoyable.\ We add unique ID assigned to each user and
    collect statistics about average time user spend.\ If user spends at least 2 hr.\ Average
    per day, we consider our system as enjoyable.
    \item \textbf{NFR02.} The system must be easy learnt.\ There is unique ID assigned to each user and
    collect the user actions statistics to the log.\ If customer ever used at least 60% of the
    total number of requirements, we consider our system to be easy learnt.
    \item \textbf{NFR03.} The system should be well organized.\ To fulfill this requirement, we follow an
    ISO 9241--161:2010 (en) Ergonomics of human-system interaction standard [\cite{iso2010ergonomics}].
    \item \textbf{NFR04.} The system should have well performance, which meant to respond it at
    least 1 second.\ User should have a device with at least 6 GB RAM and CPU with 1.8
    GHZ, 100 Mbps internet connection.\ Server must have the following hardware: Intel
    2.4 GHz 8 Cores server processor, 64GB DDR4 (4x16GB) memory, NVME or SAS
    server disk with a minimum capacity of 1.6 TB\@.
    \item \textbf{NFR05.} The unique, unambiguous identifier of users in the system is the username.
    It is set in the application’s setting.
    \item \textbf{NFR06.} The UI must be well displayed with the following browsers, in the versions
    current at the date of receipt of the system or, depending on technical possibilities,
    with the latest versions that support correct operation of the system:
    \begin{itemize}
        \item Google Chrome 72.0.36.
        \item Mozilla Firefox 64.0.2.
        \item Microsoft Edge 17.17134.
    \end{itemize}
    \item \textbf{NFR07.} The system shall force users to use passwords with a minimum length of 8
    characters and using at least one capital letter and one number.
    \item \textbf{NFR08.} The UI must be compatible to use on mobile device screens with a minimum
    width of 600 pixels.
    \item \textbf{NFR09.} The UI must be compatible to use on desktop or laptop device screens with a
    minimum display width of 1024 pixels.
\end{itemize}