\chapter{Introduction}\label{ch:introduction}
Nowadays, instant messaging systems achieve a great success and became the main mean of communication
between people via an internet.
Thanks to the simplicity and quickness of the message exchanging more and more people over the world start to use
instant messengers on daily basis.
However, such a great attention forces us to discuss another aspect of these systems, an aspect of the
information security and user privacy.
The main aim of this thesis is to design and implement an instant messaging system
that copes with the required functionalities and satisfies the defined security requirements.
More precisely, the following steps
\begin{enumerate}
    \item To provide the functional and non-functional requirements for secure instant messaging system.
    \item To eliminate security and user privacy vulnerabilities of the instant messaging system.
    \item To design and discuss secure instant messaging system.
    \begin{itemize}
        \item To propose web service architecture that fits the requirements.
        \item To propose an authorization mechanism that fits the requirements.
        \item To propose mitigations for security and user privacy vulnerabilities discussed.
        \item To design database structure.
        \item To propose planned technologies to be used during implementation of the system.
    \end{itemize}
    \item To discuss and apply E2E Encryption to the system.
    \item To design user interface that fits the designated functional requirements.
    \item To implement the software modules
    \begin{itemize}
        \item \textbf{Web Service (API).} Application Programming Interface that allows developers to create their own clients.
        Web service to be implemented using latest for the moment of writing of this thesis .NET 5 platform.
        \item \textbf{Web Client} –- Web client of the Mango Messenger.
        Web client to be implemented using Angular front-end framework with TypeScript programming language.
        \item \textbf{Mobile Client} –- Mobile client of the Mango Messenger for target platforms: Android, IOS\@.
        \item \textbf{Desktop Client} –- Desktop client of the Mango Messenger for target platforms Windows, Linux, MacOS\@.
        Desktop client to be implemented by means of existing web client and ElectronJS framework.
    \end{itemize}
\end{enumerate}