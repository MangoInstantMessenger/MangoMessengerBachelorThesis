\chapter{System requirements}\label{ch:system-requirements}

In previous sections we have briefly discussed Instant Messaging System, mainly from security and user privacy aspects.
Prior to software module implementation, it is essentially important to define the functionality module will obtain.
Not to overcomplicate this section, we discuss the secure Instant Messaging Systems requirements from customer's prospective.

There are different types of product requirements: business, functional, and non-functional.
Business requirements [\cite{dilworth2007creation}] typically answer how the product will address the needs of your company and its users.
They also reveal the business model of the app and what problems it can solve.
Functional requirements [\cite{malan2001functional}] are about functionalities that will be implemented in the app.
Non-functional requirements [\cite{chung2012non}] describe how these functionalities will be implemented.

In this article, we only focus on functional requirements.
In simple words, functional requirements are not ideas of how to solve problems or which technologies to use but rather
are descriptions of software functionality.
Mostly common and simple way to define software product's functional requirements is User Stories.
User stories [\cite{cohn2004user}] should be understandable both to developers and to you as the client, and should be written in simple words.
The most popular way of writing a user story is with the following formula:

\begin{center}
    \begin{spverbatim}
        "As a <user type>, I want <goal> so that <reason>."
    \end{spverbatim}
\end{center}


\section{Functional requirements}\label{sec:functional-requirements}
Mostly common and simple way to define software product's functional requirements are User Stories.
User stories [\cite{cohn2004user}] should be understandable both to developers and to you as the client, and should be written in simple words.
The most popular way of writing a user story is with the following formula:

\begin{center}
    \begin{spverbatim}
        "As a <user type>, I want <goal> so that <reason>."
    \end{spverbatim}
\end{center}

Now, let's group the main features of the application as follows

\begin{itemize}
    \item Registration
    \item Authentication
    \item Managing contacts
    \item Sending messages and media to individuals
    \item Creating and managing groups
    \item Sending messages and media to groups
    \item Viewing messages history
    \item Managing profile settings
\end{itemize}

In order not to overfill the document with the entire list of requirements, system's functional and non-functional
requirements moved to the annex A.


\section{Non-functional requirements}\label{sec:non-functional-requirements}
\chapter{List of non-functional requirements}\label{ch:list-of-non-functional-requirements}
\begin{itemize}
    \item \textbf{NFR01.} Graphic user interface of the system should be well organized.
    To fulfill this requirement, we follow an ISO 9241--161:2010 (en) Ergonomics of human-system interaction
    standard [\cite{iso2010ergonomics}].
    \item \textbf{NFR02.} The system should have well performance, which meant to respond it at least 1 second.
    User should have a device with at least 6 GB RAM and CPU with 1.8 GHZ, 100 Mbps internet connection.
    Server must have the following hardware: Intel 2.4 GHz 8 Cores server processor, 64GB DDR4 (4x16GB) memory, NVME or SAS
    server disk with a minimum capacity of 1.6 TB\@.
    \item \textbf{NFR03.} The unique, unambiguous identifier of users in the system is the username.
    It is set in the profile settings.
    \item \textbf{NFR04.} The UI must be well displayed with the following browsers, in the versions
    current at the date of receipt of the system or, depending on technical possibilities,
    with the latest versions that support correct operation of the system:
    \begin{itemize}
        \item Google Chrome 72.0.36.
        \item Mozilla Firefox 64.0.2.
        \item Microsoft Edge 17.17134.
    \end{itemize}
    \item \textbf{NFR05.} The system shall force users to use passwords with a minimum length of 8
    characters and using at least one capital letter and one number and one special symbol.
    \item \textbf{NFR06.} The UI must be compatible to use on mobile device screens with a minimum
    width of 600 pixels.
    \item \textbf{NFR07.} The UI must be compatible to use on desktop or laptop device screens with a
    minimum display width of 1024 pixels.
\end{itemize}


\section{Modules}\label{sec:modules}
The following modules are to be implemented:
\begin{itemize}
    \item \textbf{M01. Web Client} –- Web version of the Mango Messenger.
    \item \textbf{M02. Mobile Client} –- Mobile version of the Mango Messenger for target platforms: Android, IOS\@.
    \item \textbf{M03. Desktop Client} –- Desktop version of the Mango Messenger for target platforms Windows, Linux, MacOS\@.
    \item \textbf{M04. Web API} –- Application Programming Interface that allows developers to create their own clients.
\end{itemize}